\documentclass[12pt,a4paper]{article}

% Türkçe %
\usepackage[utf8]{inputenc} %Türkçe karakterler için
\usepackage[T1]{fontenc}
\renewcommand{\tablename}{Tablo}
\renewcommand{\figurename}{Şekil}
\renewcommand{\indexname}{Dizin}
\renewcommand{\listfigurename}{Şekiller}
\renewcommand{\listtablename}{Tablolar}
\renewcommand{\contentsname}{İçindekiler}
\setcounter{tocdepth}{3}
\setcounter{secnumdepth}{3}
% Türkçe %

\usepackage[section]{placeins}
\usepackage{geometry}
\usepackage{graphicx} %Resim koymak için
\usepackage{times} %Times fontu
\usepackage[nottoc]{tocbibind}
\usepackage{url}
\usepackage{array}
\usepackage{tabu}

\begin{document}
   \pagenumbering{gobble}
   \begin{titlepage}
   \begin{center}
      \begin{large}
         \vspace*{0.5cm}
         GAZİ ÜNİVERSİTESİ \\
         MÜHENDİSLİK FAKÜLTESİ \\
         BİLGİSAYAR MÜHENDİSLİĞİ

         \vfill
         BM 314 YAZILIM MÜHENDİSLİĞİ \\
         SPMP BELGESİ

         \vfill
         Abdullah Akalın\\Bekir Aydın\\Karim El Guermai\\Muhammed Emre Emrah\\

         \vfill
         \vspace{0.5cm}
         24.03.1017
      \end{large}
   \end{center}
\end{titlepage}

   \newpage

   \pagenumbering{roman}
   \tableofcontents
   \newpage

   \pagenumbering{arabic}

   \section{Giriş}
   Bu belgenin amacı, Kahvaltı adlı oyun projesinin fonksiyonalitesini test etmek ve kararlılığını teyid etmektir.

   \section{Kapsam}
   Belgenin kapsamı uygulamanın fonksiyonalitesinin tamamıdır. Her fonksiyon bir bir ele alınmıştır.

   \section{İlgili Belgeler}
   Bu belge daha önce tarafınıza sunmuş olduğumuz sistem gereksinimleri belgesi (SRS) ile doğrudan bağlantılı olup yazılım proje yönetim planı belgesine (SPMP) de atıflar yapılmıştır.

   \section{Takip Çizelgesi}\footnote{Bu bölümdeki gereksinimler yukarıda bahsi edilen SRS belgesinin 3.2.x bölümlerine tekabül etmektedir.}
   \begin{tabular}{ | l | l | }
   \hline
   Gereksinim & İlişkili Test \\ \hline
   \#1 & Test \#1 \\ \hline
   \#2 & Test \#2 \\ \hline
   \#3 & Test \#3 \\ \hline
   \#4 & Test \#4 \\ \hline
   \#5 & Test \#5 \\ \hline 
   \#6 & Test \#6 \\ \hline
   \#7 & Test \#7 \\ \hline
   \#8 & Test \#8 \\ \hline
   \end{tabular}

   \section{Notlar} \label{notes}
   Geliştirmiş olduğumuz proje, SPMP belgesinde belirtildiği üzere artımlı geliştirme yöntemine göre geliştirilmiştir. Bu nedenle yapılan testler her artımın sonunda gerçekleştirilmiş olup, aşağıda belirtildiği üzere \textit{başarılı} veya \textit{başarısız} olarak etiketlenmiştir. \textit{Başarılı} durumuna sahip testler, o artım esnasında yapılan test sonuçlarında gözlenen problemlerin çözüldüğünü ve son kullanıcıya sunulan uygulamada söz konusu problemlerin gözlenmeyeceğini, \textit{başarısız} durumuna sahip testler, o artım esnasında yapılan test sonucunda tespit edilip çözüme kavuşturulamayan problemlerin bulunduğu testleri göstermektedir.

   \section{Test Senaryoları} \label{test}
   Bu bölümde geliştirmiş olduğumuz oyunun gereksinimlere göre testleri ve bu testlerin açıklamaları bir tablo üzerinde gösterilmiştir.\footnote{Bu tablodaki test numaraları, belge içindeki \ref{test}.x bölümlerinde açıklanmış olan testlere tekabül etmektedir.}

   \begin{center}
      \begin{tabular}{ | l | l | l | l | }
      \hline
      Test \# & Testi Yapan & Tarih & Başarı Durumu \\ \hline
      1 & Abdullah AKALIN & 06.05.2017 & Başarısız \\ \hline
      2 & Abdullah AKALIN & 02.05.2017 & Başarılı \\ \hline
      3 & Muhammed Emre EMRAH & 18.05.2017 & Başarılı \\ \hline
      4 & Muhammed Emre EMRAH ve Karim EL GUERMAI & 25.05.2017 & Başarılı \\ \hline
      5 & Abdullah AKALIN & 18.05.2017 & Başarılı \\ \hline
      6 & Muhammed Emre EMRAH & 25.05.2017 & Başarılı \\ \hline
      7 & Bekir AYDIN & 18.05.2017 & Başarılı \\ \hline
      8 & Karim EL GUERMAI & 15.05.2017 & Başarılı \\ \hline
      \end{tabular}
   \end{center}

   \subsection{Test Elemanı: Giriş Ekranı}
   \textit{Kapsam:} Oyuncunun ekrandaki düğmeye bastığında oyunun başlaması. \newline
   \textit{Fiil:} Düğmeye dokunmak. \newline
   \textit{Ön Koşullar:} Oyununun açık olması. \newline

   \textit{Senaryo 1:} Menü ekranındayken ekranın kilitlenmesi. \newline

   \begin{center}
      \begin{tabular}{ | l | p{3cm} | p{3cm} | p{3cm} | p{5cm} | }
      \hline
      Test \# & Fiil & Girdi & Beklenen Durum & Gerçekleşen Durum \\ \hline
      1 & Ekranın kilidinin kapatılması ve tekrar açılması. & - & Bir değişikliğin olmaması. & Ekranın sol alt çeyrek kısımda çizilmesi. \\ \hline
      \end{tabular}
   \end{center}

   \subsection{Test Elemanı: Merdiven Hareketleri}
   \textit{Kapsam:} Oyuncunun ekranın sağ tarafına tıklamasıyla merdivenlerin aşağı inmesi, sol tarafına tıklayarak yukarı çıkması. \newline
   \textit{Fiil:} Ekrana dokunmak. \newline
   \textit{Ön Koşullar:} Oyununun başlamış olması. \newline

   \textit{Senaryo 1:} Ekranın sağ-sol taraflarına tıklamak. \newline

   \begin{center}
      \begin{tabular}{ | l | p{3cm} | p{3cm} | p{3cm} | p{5cm} | }
      \hline
      Test \# & Fiil & Girdi & Beklenen Durum & Gerçekleşen Durum \\ \hline
      1 & Ekranın sağ ve sol kısımlarına tıklamak.  & Dokunma hereketi. & Sağa tıklandığında merdivenlerin aşağı inmesi(dolayısıyla yukarı çıkıyor gibi görünmek), sola tıklandığında merdivenlerin aşağı inmesi. & Merdivenlerin kayması ve üst üste binmesi. \\ \hline
      \end{tabular}
   \end{center}

   \subsection{Test Elemanı: Harflerin Toplanması}
   \textit{Kapsam:} Kahve çekirdeğinin harflere çarpınca, eğer o harf ihtiyacı olan harfse onu devşirmesi. \newline
   \textit{Fiil:} Kahve çekirdeğini merdivenlerden çıkararak harfle çarpışmak. \newline
   \textit{Ön Koşullar:} Oyununun başlamış olması. \newline

   \textit{Senaryo 1:} İhtiyacı olan harf ile (tamamlaması gereken kelime için gerekli olan sıradaki harf) çarpışmak. \newline

   \begin{center}
      \begin{tabular}{ | l | p{3cm} | p{3cm} | p{3cm} | p{5cm} | }
      \hline
      Test \# & Fiil & Girdi & Beklenen Durum & Gerçekleşen Durum \\ \hline
      1 & Harf ile çarpışmak. & Ekrana dokunma hareketi. & Harfin alınması. & Harfin alınması. \\ \hline
      2 & Harf ile çarpışmak. & Ekrana dokunma hareketi. & Harfin alınması. & Sıradaki harfin boşluk olması nedeniyle harfin alınamaması. \\ \hline
      \end{tabular}
   \end{center}

   \textit{Senaryo 2:} İhtiyacı olmayan harf ile çarpışmak.

   \begin{center}
      \begin{tabular}{ | l | p{3cm} | p{3cm} | p{3cm} | p{5cm} | }
      \hline
      Test \# & Fiil & Girdi & Beklenen Durum & Gerçekleşen Durum \\ \hline
      1 & Harf ile çarpışmak. & Ekrana dokunma hareketi. & Harfin alınmaması. & Harfin alınmaması. \\ \hline
      \end{tabular}
   \end{center}

   \textit{Senaryo 3:} Alınan harfin harf tutucusunda birikmesi.

   \begin{center}
      \begin{tabular}{ | l | p{3cm} | p{3cm} | p{3cm} | p{5cm} | }
      \hline
      Test \# & Fiil & Girdi & Beklenen Durum & Gerçekleşen Durum \\ \hline
      1 & Harf ile çarpışmak. & Ekrana dokunma hareketi. & Harfin harf tutucusuna eklenmesi & Harfin harf tutucusuna küçük harflerle yerleşmesi. \\ \hline
      2 & harf ile çarpışmak. & Ekrana dokunma hareketi. & Harfin harf tutucusuna sola dayalı yerleşmesi. & Harfin harf tutucusunun ortasına yerleşmesi. \\ \hline
      \end{tabular}
   \end{center}

   \subsection{Test Elemanı: Kahve Çekirdeği ile Şekerin Çarpışması}
   \textit{Kapsam:} Kahve çekirdeğinin şekere çarpınca en son aldığı harfi düşürmesi ya da hiç harfi kalmamışsa oyunun bitmesi. \newline
   \textit{Fiil:} Kahve çekirdeğininin şekerle çarpıştırılması. \newline
   \textit{Ön Koşullar:} Oyununun başlamış olması ve sahnede şekerin belirmiş olması. \newline

   \textit{Senaryo 1:} Kahve çekirdeğinin merdivenleri çıkarken şekerle çarpışması.

   \begin{center}
      \begin{tabular}{ | l | p{3cm} | p{3cm} | p{3cm} | p{5cm} | }
      \hline
      Test \# & Fiil & Girdi & Beklenen Durum & Gerçekleşen Durum \\ \hline
      1 & Şeker ile çarpışmak. & Ekrana dokunma hareketi. & Bir harfin düşürülmesi. & Birden çok harfin düşmesi. \\ \hline
      2 & Şeker ile çarpışmak. & Ekrana dokunma hareketi. & Oyunun sonlanması. & Oyunun sonlanması. \\ \hline
      \end{tabular}
   \end{center}

   \textit{Senaryo 2:} Kahve çekirdeğinin merdivenleri inerken şekerle çarpışması.

   \begin{center}
      \begin{tabular}{ | l | p{3cm} | p{3cm} | p{3cm} | p{5cm} | }
      \hline
      Test \# & Fiil & Girdi & Beklenen Durum & Gerçekleşen Durum \\ \hline
      1 & Şeker ile çarpışmak. & Ekrana dokunma hareketi. & Bir harfin düşürülmesi. & Birden çok harfin düşmesi. \\ \hline
      2 & Şeker ile çarpışmak. & Ekrana dokunma hareketi. & Oyunun sonlanması. & Oyunun sonlanması. \\ \hline
      \end{tabular}
   \end{center}

   \subsection{Test Elemanı: Bölüm Geçme}
   \textit{Kapsam:} Bölümdeki kelime için gerekli harflerin toplanması durumunda sonraki kelimeye (aynı zamanda sonraki bölüme) geçilmesi. \newline
   \textit{Fiil:} Gerekli olan son harfin toplanması. \newline
   \textit{Ön Koşullar:} Oyununun başlamış olması ve bölümdeki kelime için gerekli harflerin sonuncusu hariç olmak üzere toplanmış olması. \newline

   \textit{Senaryo 1:} Kahve çekirdeğinin sonuncu harfi toplaması.

   \begin{center}
      \begin{tabular}{ | l | p{3cm} | p{3cm} | p{3cm} | p{5cm} | }
      \hline
      Test \# & Fiil & Girdi & Beklenen Durum & Gerçekleşen Durum \\ \hline
      1 & Harf ile çarpışmak. & Ekrana dokunma hareketi. & Harfin alınması ve bölümün geçilmesi. & Harfin alınması ve bölümün geçilmesi. \\ \hline
      \end{tabular}
   \end{center}

   \subsection{Test Elemanı: Yanma ve Oyunun Bitmesi}
   \textit{Kapsam:} Kahve çekirdeğinin kahve seline kapılması ve hiç harfi yokken şekerle çarpışması durumunda oyunu kaybetmesi. \newline
   \textit{Fiil:} Kahve seline kapılmak veya şekerle çarpışmak. \newline
   \textit{Ön Koşullar:} Oyununun başlamış olması, oyuncunun hiç harfinin olmaması veya kahve selinin çekirdeğin boyunu aşmış olması. \newline

   \textit{Senaryo 1:} Kahve çekirdeğinin hiç harfi yokken şekerle çarpışması.

   \begin{center}
      \begin{tabular}{ | l | p{3cm} | p{3cm} | p{3cm} | p{5cm} | }
      \hline
      Test \# & Fiil & Girdi & Beklenen Durum & Gerçekleşen Durum \\ \hline
      1 & Şekerle çarpışmak. & Ekrana dokunma hareketi. & Oyunun sonlanması. & Oyunun sonlanması. \\ \hline
      \end{tabular}
   \end{center}

   \textit{Senaryo 2:} Kahve çekirdeğinin kahve seli altında kalması.

   \begin{center}
      \begin{tabular}{ | l | p{3cm} | p{3cm} | p{3cm} | p{5cm} | }
      \hline
      Test \# & Fiil & Girdi & Beklenen Durum & Gerçekleşen Durum \\ \hline
      1 & Kahve selinin çekirdeğin boyunu aşması. & - & Oyunun sonlanması. & Kahve selinin çekirdeği henüz yutmadan oyunun sonlanması. \\ \hline
      \end{tabular}
   \end{center}

   \subsection{Test Elemanı: Kafein Modu}
   \textit{Kapsam:} Kafein göstergesine dokunulduğunda kafein modunun aktifleşmesi. \newline
   \textit{Fiil:} Kafein göstergesine dokunmak. \newline
   \textit{Ön Koşullar:} Oyununun başlamış olması ve kafein göstergesinde en az bir birimlik kafein bulunması. \newline

   \textit{Senaryo 1:} Kafein göstergesine dokunulması.

   \begin{center}
      \begin{tabular}{ | l | p{3cm} | p{3cm} | p{3cm} | p{5cm} | }
      \hline
      Test \# & Fiil & Girdi & Beklenen Durum & Gerçekleşen Durum \\ \hline
      1 & Göstergeye dokunmak. & Ekrana dokunma hareketi. & Kafein moduna geçilmesi ve kafein miktarının hızla azalması. & Kafein moduna geçilmesi ve kafein miktarının azalması ve bitince tekrar dolmaya başlaması. \\ \hline
      \end{tabular}
   \end{center}

   \subsection{Test Elemanı: Skorların Gösterilmesi}
   \textit{Kapsam:} Harf alımında ve kelime tamamlanmasında skorun artması ve oyun bitiminde yüksek skorla birlikte gösterilmesi. \newline
   \textit{Fiil:} Harf almak, kelime tamamlamak ya da oyunu kaybetmek. \newline
   \textit{Ön Koşullar:} Oyununun başlamış olması. \newline

   \textit{Senaryo 1:} Harf almak.

   \begin{center}
      \begin{tabular}{ | l | p{3cm} | p{3cm} | p{3cm} | p{5cm} | }
      \hline
      Test \# & Fiil & Girdi & Beklenen Durum & Gerçekleşen Durum \\ \hline
      1 & Harf almak. & Ekrana dokunma hareketi. & Puanın 1 artması. & Puanın 1 artması. \\ \hline
      \end{tabular}
   \end{center}

   \textit{Senaryo 2:} Kelime tamamlama.

   \begin{center}
      \begin{tabular}{ | l | p{3cm} | p{3cm} | p{3cm} | p{5cm} | }
      \hline
      Test \# & Fiil & Girdi & Beklenen Durum & Gerçekleşen Durum \\ \hline
      1 & Sonuncu harfi almak. & Ekrana dokunma hareketi. & Puanın 10 artması. & Puanın 11 artması. (Bu davranış tutarlıdır. Çünkü 10 puan kelime tamamlama ile +1 puan ise en son alınan harf ile gelecektir.)\\ \hline
      \end{tabular}
   \end{center}

   \section{Sonuç}
   Bu belgede projenin geliştirilmesi esnasında yapılan testler ve SRS belgesinde karşılık gelen gereksinimleri ele alınmıştır. Testlerin başarı durumları \ref{test}. bölümde belirtilmiştir. Ayrıca testlerin hangi şartlarda yapıldığı ve başarı durumlarının neleri belirttiği \ref{notes}. bölümde açıklığa kavuşturulmuştur.
   
\end{document}
